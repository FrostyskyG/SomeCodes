\documentclass{article}
\begin{document}
\section{Multiline Equations}
\subsection{Multi-line Equations}

\begin{eqnarray*}
1+2+\ldots+n &=& \frac{1}{2}((1+2+\ldots+n)+(n+\ldots+2+1)) \\
&=& \frac{1}{2}\underbrace{(n+1)+(n+1)+\ldots+(n+1)}_{\mbox{$n$ copies}}\\
&=& \frac{n(n+1)}{2} \\
\end{eqnarray*}

\subsection{Accents}
$\hat{a}, \dot{a}, \ddot{a}, \tilde{a}, \bar{a}, \vec{a}$

\subsection{Bracket Symbols}
$$
\left[
\begin{array}{cc}
1 & 2 \\
3 & 4 \\
\end{array}
\right]
$$

$$
|x| = \left\{
\begin{array}{lr}
-x & x\le 0 \\
x & x\ge 0
\end{array}
\right.
$$

\subsection{Dots}
$a_1,\ldots,a_n$

$$
\left(
\begin{array}{ccc}
a_{11} & \cdots & a_{1n} \\
\vdots & \ddots & \vdots \\
a_{m1} & \cdots & a_{mn} \\
\end{array}
\right)
$$

\subsection{Indenting}
The default for a LaTeX document is to indent new paragraphs unless the paragraph follows a section heading. If you want to change the indentation, use the indent and noindent commands respectively, at the beginning of the paragraph in question.
\\
\indent The default for a LaTeX document is to indent new paragraphs unless the paragraph follows a section heading. If you want to change the indentation, use the indent and noindent commands respectively, at the beginning of the paragraph in question.

\end{document}